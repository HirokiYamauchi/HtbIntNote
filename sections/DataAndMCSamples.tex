\section{Data and MonteCarlo Simulated Events}
\label{sec:DataAndMC}

\subsection{Data Sample}
\label{subsec:DataSample}
This analysis uses $pp$ collision data collected from 2015 to 2018 by the ATLAS detector at $\sqrt{s}=13$ TeV. Selected events are recorded using unprescaled triggers, as detailed in Table \ref{tab:GlobalLeptonTriggers}. Only runs with stable colliding beams and all ATLAS subsystems operational are used. These are summarized in the Good Run Lists (GRL) shown in Table \ref{tab:GRLForData}, together with the integrated luminosity collected each year. The total integrated luminosity is 139 $\text{fb}^{-1}$ \cite{ATLAS-CONF-2019-021}.

\begin{table}[H]
  \centering
  \subfloat[] {
    \begin{tabular*}{150mm}{@{\extracolsep{\fill}}cc}
      \hline\hline
      Year & Single-electron triggers\\
      \hline
      \multicolumn{1}{l}{2015} & e24\_lhmedium\_L1EM20VH\_OR\_e60\_lhmedium\_OR\_e120\_lhloose\\
      \multicolumn{1}{l}{2016-2018} & e26\_lhtight\_nod0\_ivarloose\_OR\_e60\_lhmedium\_nod0\_OR\_e140\_lhloose\_nod0\\
      \hline\hline
    \end{tabular*}
    \label{tab:SingleElectronTriggers}
  }\\
  \subfloat[] {
    \begin{tabular*}{100mm}{@{\extracolsep{\fill}}cc}
      \hline\hline
      Year & Single-muon triggers\\
      \hline
      \multicolumn{1}{l}{2015} & mu20\_iloose\_L1MU15\_OR\_mu50\\
      \multicolumn{1}{l}{2016-2018} & mu26\_ivarmedium\_OR\_mu50\\
      \hline\hline
    \end{tabular*}
    \label{tab:SingleMuonTriggers}
  }
  \caption{Single-electron (a) and single-muon (b) trigger menus used depending on the year of data-taking.}
  \label{tab:GlobalLeptonTriggers}
\end{table}

\begin{table}[H]
  \centering
  \begin{tabular*}{150mm}{@{\extracolsep{\fill}}ccc}
    \hline\hline
    Year & Luminosity ($\text{pb}^{-1}$) & GRL\\
    \hline
    2015 & 3219.6  & data15\_13TeV/20170619/physics\_25ns\_21.0.19.xml\\
    2016 & 32988.1 & data16\_13TeV/20180129/physics\_25ns\_21.0.19.xml\\
    2017 & 44307.4 & data17\_13TeV/20180619/physics\_25ns\_Triggerno17e33prim.xml\\
    2018 & 58450.1 & data18\_13TeV/20190318/physics\_25ns\_Triggerno17e33prim.xml\\
    \hline\hline
  \end{tabular*}
  \caption{Integrated luminosity for each year of data-taking, computed with the OflLumi-13TeV-010 luminosity
  tag \cite{LuminosityForPhysis}, together with the corresponding GRLs \cite{GoodRunListRun2}.}
  \label{tab:GRLForData}
\end{table}

\subsection{Signal Samples}
\label{subsec:SignalSample}
This paragraph describes MC samples used for each signal event's estimation. The summary is shown in Table \ref{tab:SigSampleSummary}.

\begin{table}[H]
  \centering
  \begin{tabular*}{160mm}{@{\extracolsep{\fill}}lllll}
    \hline\hline
    Physics process & Generator & PS generator & Normalisation & PDF set\\
    \hline
    $tbH^{+}$ ($M_{H^{+}}\leq3.0$ TeV)  & MG5\_aMC 2.6.2 & Pythia 8.212 & NLO & NNPDF2.3NLO\\
    $tbH^{+}$ ($M_{H^{+}}=4.0,5.0$ TeV) & MG5\_aMC 2.8.1 & Pythia 8.244 & NLO & NNPDF3.0NLO\\
    $tbW'$                              & MG5\_aMC 2.9.9 & Pythia 8.307 & NLO & NNPDF3.0NLO\\
    \hline\hline
  \end{tabular*}
  \caption{Nominal simulated signal event samples. The generator, parton shower generator and cross-section used for normalization are shown together with the applied PDF set.}
  \label{tab:SigSampleSummary}
\end{table}

\subsubsection{$\bar{t}bH^{+}$ Samples}
\label{subsec:HpSample}

\newcounter{Num}
\setcounter{Num}{2}

The $H^{+}$ signal samples are generated with MadGraph5\_aMCatNLO (MG5\_aMC) \cite{C.Degrande-2015}, which is a generator based on a four-flavor scheme (4FS) next-to-leading order (NLO) in QCD \cite{Alwall:2014hca}. The NNPDF2.3NLO \cite{Ball:2012cx} parton distribution function (PDF) set is used. \footnote{The samples with masses of 4 and 5 TeV are generated using NNPDF3.0NLO \cite{Ball:2014uwa} PDF set.} The width of the $H^{+}$ is set to zero. Dynamic QCD factorisation and renormalisation scales ($\mu_{f}$ and $\mu_{r}$) are set to $\frac{1}{3}\sum_{i}\sqrt{m(i)^{2}+p_{T}(i)^{2}}$, where $i$ runs over the final state particles ($H^{+}$, $t$ and $b$) used in the generation. The events are showered with Pythia 8.212 \cite{Sjostrand:2007gs} with the A14 \cite{ATL-PHYS-PUB-2014-021} set of underlying-event related parameters tuned to ATLAS. Ten different $H^{+}$ mass points between 1000 and 5000 GeV are generated as detailed in Table \ref{tab:HpSignalSamples}. The table also shows cross sections from MG5\_aMC and Santander-matched cross sections for 2HDM type-\Roman{Num} (a la MSSM), but without SUSY QCD corrections \cite{C.Degrande-2015, M.Flechl-2015, S.Dittmaier-2011, E.L.Berger-2005}. All samples are fully simulated with the proportions of mc16a, mc16d, and mc16e corresponding to the amount of data recorded in the 2015-2016, 2017, and 2018 data-taking years.

\begin{table}[H]
  \centering
  \begin{tabular*}{160mm}{@{\extracolsep{\fill}}cccccc}
    \hline\hline
    DSID   & $H^{+}$ mass [GeV] & Size & ${\sigma}^{\text{MG5\_aMC}}$ [fb] & ${\sigma}_{\tan{\beta}=1}^{\text{MSSM}}$ [fb] & ${\sigma}_{\tan{\beta}=60}^{\text{MSSM}}$ [fb]\\
    \hline
    450004 & 1000 & 1.0M & 3.28                  & 40.9 & 37.8\\
    450598 & 1200 & 1.0M & 1.31                  & 16.4 & 15.1\\
    450599 & 1400 & 1.0M & $5.62{\times}10^{-1}$ &  7.1 &  6.5\\
    450600 & 1600 & 1.2M & $2.54{\times}10^{-1}$ &  3.2 &  3.0\\
    450601 & 1800 & 1.3M & $1.21{\times}10^{-1}$ &  1.5 &  1.4\\
    450602 & 2000 & 1.9M & $5.90{\times}10^{-2}$ &  0.8 &  0.7\\
    451490 & 2500 & 1.9M & $1.11{\times}10^{-2}$ & \multicolumn{2}{c}{\textit{Not available}}\\
    451491 & 3000 & 1.9M & $2.34{\times}10^{-3}$ & \multicolumn{2}{c}{\textit{Not available}}\\     
    508710 & 4000 & 1.9M & $9.75{\times}10^{-5}$ & \multicolumn{2}{c}{\textit{Not available}}\\     
    508711 & 5000 & 1.9M & $4.28{\times}10^{-6}$ & \multicolumn{2}{c}{\textit{Not available}}\\     
    \hline\hline
  \end{tabular*}
  \caption{List of the generated $H^{+}$ samples. All samples are simulated with FullSim and available in the appropriate proportions of mc16a, mc16d, and mc16e. The cross-section values for $\tan{\beta}=1$ or $\tan{\beta}=60$ take into account the production of $H^{\pm}$.}
  \label{tab:HpSignalSamples}
\end{table}

\subsubsection{$\bar{t}bW'$ Samples}
\label{subsec:WpSample}
The left- and right-handed $W'$ ($W'_{L}$ and $W'_{R}$) signal samples are generated with the same options (QCD scales, PDF, NLO, and 4FS) as the $H^{+}$ signal sample generations. Nine different $W'$ mass points between 1000 and 4000 GeV are generated as same as the $H^{+}$ signal samples as detailed in Table \ref{tab:WpSignalSamples} \footnote{Only 5000 GeV mass sample aren't generated, because it is difficult technically due to its very narrow mass width.}. The table also shows cross-sections from MG5\_aMC. All samples are fully simulated with the proportions of mc16a, mc16d, and mc16e corresponding to the amount of data recorded in the 2015-2016, 2017, and 2018 data-taking years. 

\begin{table}[H]
  \centering
  \subfloat[] {
    \begin{tabular*}{130mm}{@{\extracolsep{\fill}}cccc}
        \hline\hline
        DSID   & $W'_{L}$  mass [GeV] & Size & ${\sigma}^{\text{MG5\_aMC}}$ [fb]\\
        \hline
        510889 & 1000                 & 0.5M & 22.54\\
        510890 & 1200                 & 0.5M &  8.56\\
        510891 & 1400                 & 0.5M &  3.50\\
        510892 & 1600                 & 0.5M &  1.53\\
        510893 & 1800                 & 0.5M & $7.03{\times}10^{-1}$ \\
        510894 & 2000                 & 0.5M & $3.33{\times}10^{-1}$ \\
        510895 & 2500                 & 0.5M & $5.98{\times}10^{-2}$ \\
        510896 & 3000                 & 0.5M & $1.19{\times}10^{-2}$ \\     
        510897 & 4000                 & 0.5M & $5.50{\times}10^{-4}$ \\      
        \hline\hline
    \end{tabular*}
    \label{tab:WpLSignalSamples}
  }\\
  \subfloat[] {
    \begin{tabular*}{130mm}{@{\extracolsep{\fill}}cccc}
        \hline\hline
        DSID   & $W'_{R}$ mass [GeV] & Size & ${\sigma}^{\text{MG5\_aMC}}$ [fb]\\
        \hline
        510898 & 1000                & 0.5M & 22.66 \\
        510899 & 1200                & 0.5M &  8.52 \\
        510900 & 1400                & 0.5M &  3.50 \\
        510901 & 1600                & 0.5M &  1.52 \\
        510902 & 1800                & 0.5M & $6.98{\times}10^{-1}$ \\
        510903 & 2000                & 0.5M & $3.33{\times}10^{-1}$ \\
        510904 & 2500                & 0.5M & $5.94{\times}10^{-2}$ \\
        510905 & 3000                & 0.5M & $1.19{\times}10^{-2}$ \\     
        510906 & 4000                & 0.5M & $5.48{\times}10^{-4}$ \\      
        \hline\hline
    \end{tabular*}
    \label{tab:WpRSignalSamples}
  }
  \caption{List of the generated $W'_L$ (a) and $W'_{R}$ (b) samples. All samples are simulated with FullSim and available in the appropriate proportions of mc16a, mc16d, and mc16e.}
  \label{tab:WpSignalSamples}
\end{table}


\subsection{Background Samples}
\label{subsec:BkgSample}
This paragraph describes MC samples used for each background event's estimation. The summary is shown in Table \ref{tab:BkgSampleSummary}.

\begin{table}[H]
  \centering
  \begin{tabular*}{160mm}{@{\extracolsep{\fill}}lllll}
    \hline\hline
    Physics process & Generator & PS generator & Normalisation & PDF set\\
    \hline
    $t\bar{t}+\text{jets}$ & PowhegBox v2   & Pythia 8.230 & NNLO+NNLL & NNPDF3.0NLO\\
    $t\bar{t}H$            & PowhegBox v2   & Pythia 8.230 & NNLO      & NNPDF3.0NLO\\
    $t\bar{t}V$            & MG5\_aMC 2.3.3 & Pythia 8.210 & NLO       & NNPDF3.0NLO\\
    \hline
    Single top t-chan.     & PowhegBox v2   & Pythia 8.230 & aNNLO     & NNPDF3.0NLOnf4\\
    Single top s-chan.     & PowhegBox v2   & Pythia 8.230 & aNNLO     & NNPDF3.0NLO\\
    Single top $tW$        & PowhegBox v2   & Pythia 8.230 & aNNLO     & NNPDF3.0NLO\\
    \hline
    $tHjb$                 & MG5\_aMC 2.6.X & Pythia 8.230 & NLO       & NNPDF3.0NLOnf4\\
    $tHW$                  & MG5\_aMC 2.6.2 & Pythia 8.235 & NLO       & NNPDF3.0NLO\\
    $tZq$                  & MG5\_aMC 2.3.3 & Pythia 8.212 & NLO       & CTEQ6L1LO\\
    $tZW$                  & MG5\_aMC 2.3.3 & Pythia 8.212 & NLO       & NNPDF3.0NLO\\
    4 tops                 & MG5\_aMC 2.3.3 & Pythia 8.230 & NLO       & NNPDF3.1NLO\\
    \hline
    $V+\text{jets}$        & Sherpa 2.2.1   & Sherpa 2.2.1 & NNLO      & NNPDF3.0NLO\\
    Diboson                & Sherpa 2.2     & Sherpa 2.2   & NLO       & NNPDF3.0NLO\\
    \hline\hline
  \end{tabular*}
  \caption{Nominal simulated background event samples. The generator, parton shower generator and cross-section used for normalisation are shown together with the applied PDF set.}
  \label{tab:BkgSampleSummary}
\end{table}


%--- ttbar+jets
\subsubsection{$t\bar{t}$+jets}
\label{subsec:TtbarSamples}
The production of $t\bar{t}$ events is modeled using the PowhegBox \cite{Nason:2004rx, Frixione:2007vw, Alioli:2010xd, J.M.Campbell-2015} v2 generator, which provides matrix element (ME) at NLO in the strong coupling constant ($\alpha_{S}$) with the NNPDF3.0NLO PDF set \cite{Ball:2014uwa} and the $h_{\text{damp}}$ parameter \footnote{The $h_{\text{damp}}$ parameter controls the transverse momentum of the first additional emission beyond the LO Feynman diagram in the parton shower and therefore regulates the high-$p_{\text{T}}$ emission against which the $t\bar{t}$ system recoils.} set to $1.5m_\text{{top}}$ \cite{ATL-PHYS-PUB-2016-020}. The functional form of $\mu_{f}$ and $\mu_{r}$ is set to the default scale $\sqrt{m_{t}^{2}+p_{\text{T},t}^{2}}$. The events are showered with Pythia 8.230 \cite{Sjostrand:2014zea}.

\vskip.2\baselineskip

The uncertainty due to initial-state-radiation (ISR) is estimated using weights in the ME and in the parton shower (PS). To simulate higher parton radiation $\mu_{f}$ and $\mu_{r}$ are varied by a factor of 0.5 in the ME while using the \textit{Var3c} upward variation from the A14 tune. For lower parton radiation, $\mu_{f}$ and $\mu_{r}$ varied by a factor of 2.0 while using the \textit{Var3c} downward variation in the PS. The impact of final-state-radiation (FSR) is evaluated using PS weights which vary $\mu_{r}$ for QCD emission in the FSR by a factor of 0.5 and 2.0, respectively. The impact of the PS and hadronisation model is evaluated by changing the showering of the nominal PowhegBox events from Pythia to Herwig 7.04 \cite{Bahr:2008pv, Bellm:2015jjp}.

\vskip.2\baselineskip

To assess the uncertainty due to the choice of the matching scheme, the Powheg sample is compared to a sample of events generated with MG5\_aMC v2.6.0 and the NNPDF3.0NLO PDF set showered with Pythia 8.230. The shower starting scale has the functional form $\mu_{\text{q}}=H_{\text{T}}/2$ \cite{ATL-PHYS-PUB-2017-007}, where $H_\text{T}$ is defined as the scalar sum of the $p_{\text{T}}$ of all outgoing partons. Choice of $\mu_{f}$ and $\mu_{r}$ is the same as that for the Powheg setup.

\vskip.2\baselineskip

To enhance the statistics in the phase-space relevant for this analysis, for all the samples described above, dedicated filtered samples were produced, requiring $b$- or $c$-hadrons in addition to those arising from the decays of the top quarks, as follows:

\begin{itemize}
  \item One sample was produced with at least two additional $b$-hadrons with $p_{\text{T}}>15$ GeV.
  \item One sample was produced with at least one additional $b$-hadron with $p_{\text{T}}>5$ GeV and failing the previous requirement.
  \item One sample was produced with at least one additional $c$-hadron with $p_{\text{T}}>15$ GeV and failing the previous two requirements.
\end{itemize}

The combined use of the unfiltered and filtered samples is done by assuring no overlap between them (by the use of the heavy flavour filter flag, \textit{TopHeavyFlavorFilterFlag}) and weighted with the appropriate cross-section and filter efficiencies.


%--- ttH
\subsubsection{$t\bar{t}H$}
\label{subsec:TthSamples}

The production of $t\bar{t}H$ events is modeled in the 5F scheme using PowhegBox \cite{Hartanto:2015uka} at NLO in $\alpha_{S}$ with the NNPDF3.0NLO PDF set. The $h_{\text{damp}}$ parameter is set to $3/4\times(m_{t}+m_{\bar{t}}+m_{H})=352.5$ GeV. The events are showered with Pythia 8.230. The uncertainties due to ISR, FSR, PS and hadronisation model, as well as that due to the matching scheme, are evaluated with the same procedures used for the $t\bar{t}+\text{jets}$ background.


%--- ttV
\subsubsection{$t\bar{t}V$}
\label{subsec:TtvSamples}

The production of $t\bar{t}V$ events is modeled using the MG5\_aMC v2.3.3 generator, which provides ME at NLO in $\alpha_{S}$ with the NNPDF3.0NLO PDF set. The functional form of $\mu_{f}$ and $\mu_{r}$ is set to the default scale $0.5{\times}{\sum_{i}}\sqrt{m_{i}^{2}+p_{\text{T},i}^{2}}$ where the sum runs over all the particles generated from the ME calculation. The events are showered with Pythia 8.210.

\vskip.2\baselineskip

Additional $t\bar{t}V$ samples are produced with Sherpa 2.2.0 \cite{Bothmann:2019yzt} at LO accuracy, using the MEPS@LO setup \cite{Catani:2001cc, Hoeche:2009rj} with up to one additional parton for the $t\bar{t}V$ sample and two additional partons for the others. A dynamic $\mu_{r}$ is used, defined similarly to that of the nominal MG5\_aMC+Pythia samples. The CKKW matching scale of the additional emissions is set to 30 GeV. The default Sherpa 2.2.0 PS is used along with the NNPDF3.0NNLO PDF set.


%--- Single top
\subsubsection{Single top}
\label{subsec:SingletopSamples}

\begin{description}
  %--- t-channel
  \item[$t$-channel] \mbox{}\\
    Single-top $t$-channel production is modeled using the PowhegBox v2 generator, which provides ME at NLO in $\alpha_{\text{S}}$ in the 4F scheme with the NNPDF3.0NLOnf4 PDF set. The functional form of $\mu_{f}$ and $\mu_{r}$ is set to $\sqrt{m_{b}^{2}+p_{\text{T},b}^{2}}$, following the recommendation of Ref.~\cite{Frederix:2012dh}. The events are showered with Pythia 8.230.
    \vskip.2\baselineskip
    The impact of the PS and hadronisation model is evaluated by comparing the nominal generator setup with a sample produced with the PowhegBox v2 generator at NLO in QCD in the 4FS using the NNPDF3.0NLOnf4 PDF set. The same events produced for the nominal PowhegBox+Pythia8 sample are used. The events are showered with Herwig 7.04.
    \vskip.2\baselineskip
    To assess the uncertainty due to the choice of the matching scheme, the nominal sample is compared to a sample generated with the MG5\_aMC v2.6.2 generator at NLO in QCD in the 4FS, using the NNPDF3.0NLOnf4 PDF set. Top quarks are decayed at LO using MadSpin \cite{Frixione:2007zp, Artoisenet:2012st} to preserve all spin correlations. The events are showered with Pythia 8.230.
  %--- s-channel    
  \item[$s$-channel] \mbox{}\\
    Single-top $s$-channel production is modeled using the PowhegBox v2 generator, which provides ME at NLO in $\alpha_{\text{S}}$ in the 5F scheme with the NNPDF3.0NLO PDF set. The functional form of $\mu_{f}$ and $\mu_{r}$ is set to the default scale, which is equal to the top quark mass. The events are showered with Pythia 8.230.
    \vskip.2\baselineskip
    The impact of the PS and hadronisation model is evaluated by comparing the nominal generator setup with a sample produced with the PowhegBox v2 generator at NLO in QCD in the 5FSusing the NNPDF3.0NLO PDF set. The same events produced for the nominal PowhegBox+Pythia8 sample are used. The events are showered with Herwig 7.04.
    \vskip.2\baselineskip
    To assess the uncertainty due to choice of the matching scheme, the nominal sample is compared to a sample generated with the MG5\_aMC v2.6.2 generator at NLO in QCD in the 5FS, using the NNPDF3.0NLO PDF set. Top quarks are decayed at LO using MadSpin to preserve all spin correlations. The events are showered with Pythia 8.230.
  %--- tW-channel
  \item[$tW$] \mbox{}\\
    Single-top $tW$ associated production is modeled using the PowhegBox v2 generator, which provides ME at NLO in $\alpha_{\text{S}}$ in the 5F scheme with the NNPDF3.0NLO PDF set. The functional form of $\mu_{f}$ and $\mu_{r}$ is set to the default scale, which is equal to the top quark mass. The diagram removal scheme \cite{Frixione:2008yi} is employed to handle the interference with $t\bar{t}$  production \cite{ATL-PHYS-PUB-2016-020}. The events are showered with Pythia 8.230.
    \vskip.2\baselineskip
    The nominal Powheg+Pythia8 sample is compared to an alternative sample generated using the diagram subtraction scheme \cite{ATL-PHYS-PUB-2016-020, Frixione:2008yi} to estimate the uncertainty due to the interference with $t\bar{t}$ production.
    \vskip.2\baselineskip
    The impact of the PS and hadronisation model is evaluated by comparing the nominal generator setup with a sample produced with the Powheg v2 generator at NLO in QCD in the 5FS using the NNPDF3.0NLO PDF set. The same events produced for the nominal Powheg+Pythia8 sample are used. The events are showered with Herwig7.04.
    \vskip.2\baselineskip
    To assess the uncertainty due to the choice of the matching scheme, the nominal sample is compared to a sample generated with the MG5\_aMC v2.6.2 generator at NLO in QCD in the 5FS, using the NNPDF2.3NLO PDF set. The events are showered with Pythia 8.230.
\end{description}


%--- tH
\subsubsection{$tH$}
\label{subsec:ThSamples}

\begin{description}
  %--- tHjb SM production
  \item[$tHjb$ production] \mbox{}\\
    The production of $tHjb$ events is modeled in the 4F scheme using the MG5\_aMCv2.6.0 with the NNPDF3.0NLOnf4 PDF set. The functional form of $\mu_{f}$ and $\mu_{r}$ is set to the default scale $1/2\times\sum_{i}\sqrt{m_{i}^{2}+p_{\text{T},i}^{2}}$, where the sum runs over all the particles generated from the ME calculation. The shower starting scale has the functional form $\mu_{q}=H_{T}/2$, where $H_{T}$ is defined as the scalar sum of the $p_{\text{T}}$ of all outgoing partons. The events are showered with Pythia 8.230.
  %--- tHW SM production
  \item[$tHW$ production] \mbox{}\\
    The production of $tHW$ events is modeled in the 5F scheme using the MG5\_aMCv2.6.2 with the NNPDF3.0NLO PDF set. The functional form $\mu_{f}$ and $\mu_{r}$ is set to the default scale $1/2\times\sum_{i}\sqrt{m_{i}^{2}+p_{\text{T},i}^{2}}$ where the sum runs over all the particles generated from the ME calculation. The shower starting scale has the functional form $\mu_{q}=H_{T}/2$, where $H_{T}$ is defined as the scalar sum of the $p_{\text{T}}$ of all outgoing partons. The events are showered with Pythia 8.235.
\end{description}


%--- Rare t processes
\subsubsection{Rare $t$ processes}
\label{subsec:RareTopSamples}

\begin{description}
  %--- tZq
  \item[$tZq$] \mbox{}\\
    The $tZq$ MC samples \cite{TOPQ-2016-14} are generated at LO in $\alpha_{\text{S}}$ using MG5\_aMC 2.3.3 in the 4F scheme, with the CTEQ6L1 \cite{Pumplin:2002vw} LO PDF set. Following the recommendations taken from Ref.~\cite{Frederix:2012dh}, the renormalisation and factorisation scales are set to $4\times\sum_{b}\sqrt{m_{i}^{2}+p_{\text{T},b}^{2}}$, where the $b$-quark is the one coming from the gluon splitting. The events are showered with Pythia 8.212.
  %--- tZW
  \item[$tZW$] \mbox{}\\
    The $tZW$ sample is simulated using the MG5\_aMC v2.3.3 generator at NLO in $\alpha_{\text{S}}$ with the NNPDF3.0NLO PDF set. The top quark is decayed inclusively while the $Z$ boson decays to a pair of leptons, by means of Pythia 8.212. The 5F scheme is used where all the quark masses are set to zero, except for the top quark. $\mu_{f}$ and $\mu_{r}$ are set to the top quark mass. The DR1 scheme \cite{Frixione:2008yi} is employed to handle the interference between $tWZ$ and $ttZ$, and is applied to the $tWZ$ sample.
    %--- 4 tops
  \item[4 tops] \mbox{}\\
    The production of 4 tops events is modeled using the MG5\_aMC v2.3.3 generator, which provides ME at NLO in $\alpha_{\text{S}}$ with the NNPDF3.1NLO PDF set. The functional form of $\mu_{f}$ and $\mu_{r}$ is set to $0.25\times\sum_{i}\sqrt{m_{i}^{2}+p_{\text{T},i}^{2}}$, where the sum runs over all the particles generated from the ME calculation, following the Ref.\cite{Frederix:2017wme}. The events are showered with Pythia 8.230.
\end{description}


%--- V+jets
\subsubsection{Vector bosons plus jets}
\label{subsec:VplusJetsSamples}

QCD vector bosons plus jets production is simulated with the Sherpa v2.2.1 PS Monte Carlo generator. In this setup, NLO-accurate ME for up to two jets, and LO-accurate ME for up to four jets are calculated with the Comix \cite{Gleisberg:2008fv} and OpenLoops \cite{Cascioli:2011va, Denner:2016kdg} libraries. The default Sherpa PS \cite{Schumann:2007mg} based on Catani-Seymour dipoles and the cluster hadronisation model \cite{Winter:2003tt} are used. They employ the dedicated set of tuned parameters developed by the Sherpa authors for this version based on the NNPDF3.0nnlo set. The NLO ME of a given jet-multiplicity are matched to the PS using a colour-exact variant of the MC@NLO algorithm \cite{Hoeche:2011fd}. Different jet multiplicities are then merged into an inclusive sample using an improved CKKW matching procedure \cite{Catani:2001cc, Hoeche:2009rj}, which is extended to NLO accuracy using the MEPS@NLO prescription \cite{Hoeche:2012yf}. The merging cut is set to $Q_\text{cut}=20$ GeV.

\vskip.2\baselineskip

QCD scale uncertainties are evaluated on-the-fly \cite{Bothmann:2016nao} using 7-point variations of $\mu_{f}$ and $\mu_{r}$ in the ME. The scales are varied independently by factors of 0.5 and 2 but avoiding opposite factors. PDF uncertainties for the nominal PDF set are evaluated using the 100 variation replicas, as well as $\pm{0.001}$ shifts of $\alpha_{\text{S}}$.


%--- VV
\subsubsection{Dibosons}
\label{subsec:DibosonSamples}

Diboson samples are simulated with the Sherpa v2.2 generator. In this setup multiple ME are matched and merged with the Sherpa PS based on Catani-Seymour dipole using the MEPS@NLO prescription. For semileptonically and fully leptonically decaying diboson samples, as well as loop-induced diboson samples, the virtual QCD correction for ME at NLO accuracy are provided by the OpenLoops library. For electroweak $VVjj$ production, the calculation is performed in the $G_{\mu}$ scheme, ensuring an optimal description of pure electroweak interactions at the electroweak scale. All samples are generated using the NNPDF3.0nnlo set, along with the dedicated set of tuned PS parameters developed by the Sherpa authors.


%% %--- Summary
%% \subsubsection{Summary}
%% \label{subsec:SummarySamples}

%% The samples and their basic generation parameters are summarized in Table \ref{tab:SampleSummary}. Exact dataset names of TOPQ1 DAODs are shown in App. \ref{app:TOPQ1DAODList}

%% \begin{table}[H]
%%   \centering
%%   \begin{tabular*}{160mm}{@{\extracolsep{\fill}}lllll}
%%     \hline\hline
%%     Physics process & Generator & PS generator & Normalisation & PDF set\\
%%     \hline
%%     $tbH^{+}$ ($M_{H^{+}}\leq3.0$ TeV)  & MG5\_aMC 2.6.2 & Pythia 8.212 & NLO & NNPDF2.3NLO\\
%%     $tbH^{+}$ ($M_{H^{+}}=4.0,5.0$ TeV) & MG5\_aMC 2.8.1 & Pythia 8.244 & NLO & NNPDF3.0NLO\\
%%     \hline
%%     $t\bar{t}+\text{jets}$ & PowhegBox v2   & Pythia 8.230 & NNLO+NNLL & NNPDF3.0NLO\\
%%     $t\bar{t}H$            & PowhegBox v2   & Pythia 8.230 & NNLO      & NNPDF3.0NLO\\
%%     $t\bar{t}V$            & MG5\_aMC 2.3.3 & Pythia 8.210 & NLO       & NNPDF3.0NLO\\
%%     \hline
%%     Single top t-chan. & PowhegBox v2 & Pythia 8.230 & aNNLO & NNPDF3.0NLOnf4\\
%%     Single top s-chan. & PowhegBox v2 & Pythia 8.230 & aNNLO & NNPDF3.0NLO\\
%%     Single top $tW$    & PowhegBox v2 & Pythia 8.230 & aNNLO & NNPDF3.0NLO\\
%%     \hline
%%     $tHjb$ & MG5\_aMC 2.6.X & Pythia 8.230 & NLO & NNPDF3.0NLOnf4\\
%%     $tHW$  & MG5\_aMC 2.6.2 & Pythia 8.235 & NLO & NNPDF3.0NLO\\
%%     $tZq$  & MG5\_aMC 2.3.3 & Pythia 8.212 & NLO & CTEQ6L1LO\\
%%     $tZW$  & MG5\_aMC 2.3.3 & Pythia 8.212 & NLO & NNPDF3.0NLO\\
%%     4 tops & MG5\_aMC 2.3.3 & Pythia 8.230 & NLO & NNPDF3.1NLO\\
%%     \hline
%%     $V+\text{jets}$ & Sherpa 2.2.1 & Sherpa 2.2.1 & NNLO & NNPDF3.0NLO\\
%%     Diboson         & Sherpa 2.2   & Sherpa 2.2   & NLO  & NNPDF3.0NLO\\
%%     \hline\hline
%%   \end{tabular*}
%%   \caption{Nominal simulated signal and background event samples. The generator, parton shower generator and cross-section used for normalisation are shown together with the applied PDF set.}
%%   \label{tab:SampleSummary}
%% \end{table}

