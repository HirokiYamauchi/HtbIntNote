\newpage
%-------------------------------------------------------------------------------
\section{\File{atlasjournal.sty}}

Turn on including these definitions with the option \Option{journal=true} and off with the option \Option{journal=false}.

\begin{xtabular}{ll}
\verb|\AcPA| & \AcPA \\
\verb|\ARevNS| & \ARevNS \\
\verb|\CPC| & \CPC \\
\verb|\EPJ| & \EPJ \\
\verb|\EPJC| & \EPJC \\
\verb|\FortP| & \FortP \\
\verb|\IJMP| & \IJMP \\
\verb|\JETP| & \JETP \\
\verb|\JETPL| & \JETPL \\
\verb|\JaFi| & \JaFi \\
\verb|\JHEP| & \JHEP \\
\verb|\JMP| & \JMP \\
\verb|\MPL| & \MPL \\
\verb|\NCim| & \NCim \\
\verb|\NIM| & \NIM \\
\verb|\NIMA| & \NIMA \\
\verb|\NP| & \NP \\
\verb|\NPB| & \NPB \\
\verb|\PL| & \PL \\
\verb|\PLB| & \PLB \\
\verb|\PR| & \PR \\
\verb|\PRC| & \PRC \\
\verb|\PRD| & \PRD \\
\verb|\PRL| & \PRL \\
\verb|\PRep| & \PRep \\
\verb|\RMP| & \RMP \\
\verb|\ZfP| & \ZfP \\
\verb|\collab| & \collab \\
\end{xtabular}



\newpage
%-------------------------------------------------------------------------------
\section{\File{atlasmisc.sty}}

Turn on including these definitions with the option \Option{misc=true} and off with the option \Option{misc=false}.

%-------------------------------------------------------------------------------
% Miscellaneous useful definitions for ATLAS documents.
% Include with misc option in atlasphysics.sty.
%
% Note that this file can be overwritten when atlaslatex is updated.
%
% Copyright (C) 2002-2020 CERN for the benefit of the ATLAS collaboration
%-------------------------------------------------------------------------------

% +--------------------------------------------------------------------+
% |  Useful things for proton-proton physics                           |
% +--------------------------------------------------------------------+
\newcommand*{\pT}{\ensuremath{p_{\text{T}}}\xspace}
\newcommand*{\pt}{\ensuremath{p_{\text{T}}}\xspace}
\newcommand*{\ET}{\ensuremath{E_{\text{T}}}\xspace}
\newcommand*{\eT}{\ensuremath{E_{\text{T}}}\xspace}
\newcommand*{\et}{\ensuremath{E_{\text{T}}}\xspace}
\newcommand*{\HT}{\ensuremath{H_{\text{T}}}\xspace}
\newcommand*{\pTsq}{\ensuremath{p_{\text{T}}^{2}}\xspace}
%\newcommand*{\ptsq}{\ensuremath{p_{\text{T}}^{2}}\xspace}
\newcommand*{\MET}{\ensuremath{E_{\text{T}}^{\text{miss}}}\xspace}
\newcommand*{\met}{\ensuremath{E_{\text{T}}^{\text{miss}}}\xspace}
\newcommand*{\sumET}{\ensuremath{\sum \ET}\xspace}
\newcommand*{\EjetRec}{\ensuremath{E_{\text{rec}}}\xspace}
\newcommand*{\PjetRec}{\ensuremath{p_{\text{rec}}}\xspace}
\newcommand*{\EjetTru}{\ensuremath{E_{\text{true}}}\xspace}
\newcommand*{\PjetTru}{\ensuremath{p_{\text{true}}}\xspace}
\newcommand*{\EjetDM}{\ensuremath{E_{\text{DM}}}\xspace}
\newcommand*{\Rcone}{\ensuremath{R_{\text{cone}}}\xspace}
\newcommand*{\abseta}{\ensuremath{|\eta|}\xspace}
\newcommand*{\Ecm}{\ensuremath{E_{\text{cm}}}\xspace}
\newcommand*{\rts}{\ensuremath{\sqrt{s}}\xspace}
\newcommand*{\sqs}{\ensuremath{\sqrt{s}}\xspace}
\newcommand*{\Nevt}{\ensuremath{N_{\mathrm{evt}}}\xspace}
\newcommand*{\zvtx}{\ensuremath{z_{\mathrm{vtx}}}\xspace}
\newcommand*{\dzero}{\ensuremath{d_{0}}\xspace}
\newcommand*{\zzsth}{\ensuremath{z_{0} \sin(\theta)}\xspace}

% LHC standard terms
\newcommand*{\RunOne}{Run~1\xspace}
\newcommand*{\RunTwo}{Run~2\xspace}
\newcommand*{\RunThr}{Run~3\xspace}

% ATLAS jets standard terms
\newcommand*{\kt}{\ensuremath{k_{t}}\xspace}
\newcommand*{\antikt}{anti-\kt}
\newcommand*{\Antikt}{Anti-\kt}
% Different hyphenation for pile-up in UK and US English
\iflanguage{USenglish}{%
    \newcommand*{\pileup}{pileup\xspace}
    \newcommand*{\Pileup}{Pileup\xspace}
}{%
    \newcommand*{\pileup}{pile-up\xspace}
    \newcommand*{\Pileup}{Pile-up\xspace}
}

% b-tagging standard terms
\newcommand*{\btag}{\ensuremath{b\text{-tagging}}\xspace}
\newcommand*{\btagged}{\ensuremath{b\text{-tagged}}\xspace}
\newcommand*{\bquark}{\ensuremath{b\text{-quark}}\xspace}
\newcommand*{\bquarks}{\ensuremath{b\text{-quarks}}\xspace}
\newcommand*{\bjet}{\ensuremath{b\text{-jet}}\xspace}
\newcommand*{\bjets}{\ensuremath{b\text{-jets}}\xspace}


% +--------------------------------------------------------------------+
% | Masses                                                             |
% +--------------------------------------------------------------------+
\newcommand*{\mh}{\ensuremath{m_{h}}\xspace}
\newcommand*{\mW}{\ensuremath{m_{W}}\xspace}
\newcommand*{\mZ}{\ensuremath{m_{Z}}\xspace}
\newcommand*{\mH}{\ensuremath{m_{H}}\xspace}
% \newcommand*{\mA}{\ensuremath{m_{A}}\xspace}

%\def\mass#1{\ensuremath{m_{#1#1}}}%  "\mass{\mu}" produces "msub{mumu}".
\newcommand{\twomass}[2]{\ensuremath{m_{#1#2}}\xspace}

% +--------------------------------------------------------------------+
% |  Monte Carlo generators                                            |
% +--------------------------------------------------------------------+
\newcommand*{\ACERMC}{\textsc{AcerMC}\xspace}
\newcommand*{\ACERMCV}[1]{\textsc{AcerMC}~#1\xspace}
\newcommand*{\ALPGEN}{\textsc{Alpgen}\xspace}
\newcommand*{\ALPGENV}[1]{\textsc{Alpgen}~#1\xspace}
\newcommand*{\COLLIER}{\textsc{Collier}\xspace}
\newcommand*{\EVTGEN}{\textsc{EvtGen}\xspace}
\newcommand*{\GEANT}{\textsc{Geant}\xspace}
\newcommand*{\GGTOVV}{\textsc{gg2VV}\xspace}
\newcommand*{\GOSAM}{\textsc{GoSam}\xspace}
\newcommand*{\Herwigpp}{Herwig++\xspace}
\newcommand*{\HERWIGpp}{Herwig++\xspace}
\newcommand*{\Herwig}{Herwig\xspace}
\newcommand*{\HerwigV}[1]{Herwig~#1\xspace}
\newcommand*{\HATHOR}{\textsc{Hathor}\xspace}
\newcommand*{\HERWIG}{\textsc{Herwig}\xspace}
\newcommand*{\HERWIGV}[1]{\textsc{Herwig}~#1\xspace}
\newcommand*{\JIMMY}{\textsc{Jimmy}\xspace}
\newcommand*{\JIMMYV}[1]{\textsc{Jimmy}~#1\xspace}
\newcommand*{\MADSPIN}{\textsc{MadSpin}\xspace}
\newcommand*{\MADGRAPH}{\textsc{MadGraph}\xspace}
\newcommand*{\MADGRAPHV}[1]{\textsc{MadGraph}~#1\xspace}
\newcommand*{\MEPSatLO}{\textsc{MEPS@LO}\xspace}
\newcommand*{\MEPSatNLO}{\textsc{MEPS@NLO}\xspace}
\newcommand*{\MGMCatNLO}{\textsc{MadGraph5}\_aMC@NLO\xspace}
\newcommand*{\MGMCatNLOV}[1]{\textsc{MadGraph5}\_aMC@NLO~#1\xspace}
\newcommand*{\MCatNLO}{MC@NLO\xspace}
\newcommand*{\MCatNLOV}[1]{MC@NLO~#1\xspace}
\newcommand*{\MINLO}{MiNLO\xspace}
\newcommand*{\AMCatNLO}{aMC@NLO\xspace}
\newcommand*{\AMCatNLOV}[1]{aMC@NLO~#1\xspace}
\newcommand*{\MCFM}{MCFM\xspace}
\newcommand*{\MCFMV}[1]{MCFM~#1\xspace}
\newcommand*{\METOP}{\textsc{MEtop}\xspace}
\newcommand*{\METOPV}[1]{\textsc{MEtop}~#1\xspace}
\newcommand*{\OPENLOOPSV}[1]{\textsc{OpenLoops}~#1\xspace}
\newcommand*{\POWHEG}{\textsc{Powheg}\xspace}
\newcommand*{\POWHEGV}[1]{\textsc{Powheg}~#1\xspace}
\newcommand*{\POWHEGBOX}{\textsc{Powheg-Box}\xspace}
\newcommand*{\POWHEGBOXV}[1]{\textsc{Powheg-Box}~#1\xspace}
\newcommand*{\POWPYTHIA}{\POWHEG{}+\PYTHIA}
\newcommand*{\PHOTOSPP}{\textsc{Photos}++\xspace}
\newcommand*{\PROTOS}{\textsc{Protos}\xspace}
\newcommand*{\PYTHIA}{\textsc{Pythia}\xspace}
\newcommand*{\PYTHIAV}[1]{\textsc{Pythia}~#1\xspace}
\newcommand*{\SHERPA}{\textsc{Sherpa}\xspace}
\newcommand*{\SHERPAV}[1]{\textsc{Sherpa}~#1\xspace}
\newcommand*{\VBFNLO}{\textsc{VBFNLO}\xspace}

%%% Event generator extras
\newcommand*{\AUET}{AUET2\xspace}
\newcommand*{\AZNLO}{AZNLO\xspace}
\newcommand*{\Comphep}{CompHEP\xspace}
\newcommand*{\FXFX}{\textsc{FxFx}\xspace}
\newcommand*{\Monash}{Monash\xspace}
\newcommand*{\NLOEWvirt}{NLO~\ensuremath{\text{EW}_\text{virt}}\xspace}
\newcommand*{\Perugia}{Perugia\xspace}
\newcommand*{\Prospino}{Prospino\xspace}
\newcommand*{\UEEE}{UE-EE-5\xspace}

\newcommand*{\LO}{\ensuremath{\text{LO}}\xspace}
\newcommand*{\NLO}{\ensuremath{\text{NLO}}\xspace}
\newcommand*{\NLL}{\ensuremath{\text{NLL}}\xspace}
\newcommand*{\NNLO}{\ensuremath{\text{N}}\NLO\xspace}
\newcommand*{\muF}{\ensuremath{\mu_\mathrm{f}}\xspace}
\newcommand*{\muQ}{\ensuremath{\mu_\mathrm{q}}\xspace}
\newcommand*{\muR}{\ensuremath{\mu_\mathrm{r}}\xspace}
\newcommand*{\hdamp}{\ensuremath{h_\mathrm{damp}}\xspace}

% +--------------------------------------------------------------------+
% |  Useful symbols for use in or out of math mode                     |
% +--------------------------------------------------------------------+
\newcommand*{\ra}{\ensuremath{\rightarrow}\xspace}
\newcommand*{\la}{\ensuremath{\leftarrow}\xspace}
\newcommand*{\rarrow}{\ensuremath{\rightarrow}\xspace}
\newcommand*{\larrow}{\ensuremath{\leftarrow}\xspace}
%\let\rarrow=\ra
%\let\larrow=\la
\newcommand*{\lapprox}{\ensuremath{\sim\kern-1em\raise 0.65ex\hbox{\(<\)}}\xspace}%  Or use \lsim
\newcommand*{\rapprox}{\ensuremath{\sim\kern-1em\raise 0.65ex\hbox{\(>\)}}\xspace}%  and \rsim.
\newcommand*{\gam}{\ensuremath{\gamma}\xspace}
\newcommand*{\stat}{\mbox{\(\;\)(stat.)}\xspace}
\newcommand*{\syst}{\mbox{\(\;\)(syst.)}\xspace}
\newcommand*{\radlength}{\ensuremath{X_0}}
\newcommand*{\StoB}{\ensuremath{S/B}\xspace}

% Different differential symbols in American and British English
\iflanguage{USenglish}{%
  \providecommand*{\dif}{\ensuremath{d}}
}{%
  \providecommand*{\dif}{\ensuremath{\mathrm{d}}}
}


% +--------------------------------------------------------------------+
% |  QCD                                                               |
% +--------------------------------------------------------------------+
\newcommand*{\alphas}{\ensuremath{\alpha_{\text{S}}}\xspace}
\newcommand*{\NF}{\ensuremath{N_{\text{F}}}\xspace}
\newcommand*{\NC}{\ensuremath{N_{\text{C}}}\xspace}
\newcommand*{\CF}{\ensuremath{C_{\text{F}}}\xspace}
\newcommand*{\CA}{\ensuremath{C_{\text{A}}}\xspace}
\newcommand*{\TF}{\ensuremath{T_{\text{F}}}\xspace}
\newcommand*{\Lms}{\ensuremath{\Lambda_{\overline{\text{MS}}}}\xspace}
\newcommand*{\Lmsfive}{\ensuremath{\Lambda^{(5)}_{\overline{\text{MS}}}}\xspace}
\newcommand*{\kperp}{\ensuremath{k_{\perp}}\xspace}

% +--------------------------------------------------------------------+
% |  CKM matrix                                                        |
% +--------------------------------------------------------------------+
\newcommand*{\Vcb}{\ensuremath{\vert V_{cb} \vert}\xspace}
\newcommand*{\Vub}{\ensuremath{\vert V_{ub} \vert}\xspace}
\newcommand*{\Vtd}{\ensuremath{\vert V_{td} \vert}\xspace}
\newcommand*{\Vts}{\ensuremath{\vert V_{ts} \vert}\xspace}
\newcommand*{\Vtb}{\ensuremath{\vert V_{tb} \vert}\xspace}
\newcommand*{\Vcs}{\ensuremath{\vert V_{cs} \vert}\xspace}
\newcommand*{\Vud}{\ensuremath{\vert V_{ud} \vert}\xspace}
\newcommand*{\Vus}{\ensuremath{\vert V_{us} \vert}\xspace}
\newcommand*{\Vcd}{\ensuremath{\vert V_{cd} \vert}\xspace}

% +--------------------------------------------------------------------+
% |  Figure width                                                      |
% +--------------------------------------------------------------------+
\newlength{\figwidth}
\setlength{\figwidth}{\textwidth}
\addtolength{\figwidth}{-2.0cm}

% +--------------------------------------------------------------------+
% |  The decay symbol, to be used in \eqalign.                         |
% |  It works like: \[\eqalign{a\ra &b+c\cr &\dk &e+f\cr &&\dk g+h}\]  |
% |                                                                    |
% |                  a  -->  b + c                                     |
% |                          |                                         |
% |                          |                                         |
% |                          +----> e + f                              |
% |                                 |                                  |
% |                                 |                                  |
% |                                 +----> g + h                       |
% +--------------------------------------------------------------------+
\newdimen\dkwidth
\def\dk{%
   \dkwidth=\baselineskip
   {\def\to{\rightarrow}%  allows "\rightarrowfill" to work.
   \kern 3pt
   \hbox{%
      \raise 3pt
      \hbox{%
         \vrule height 0.8\dkwidth width 0.7pt depth0pt
      }
      \kern-0.4pt%
      \hbox to 1.5\dkwidth{%
         \rightarrowfill
      }
   \kern0.6em
   }}
}

% +--------------------------------------------------------------------+
% |  Redefine \eqalign to allow more than one column; very             |
% |  useful for multiple decays as defined above.                      |
% +--------------------------------------------------------------------+
%\unlock
\def\eqalign#1{%
   \,
   \vcenter{%
      \openup\jot\m@th
      \ialign{%
         \strut\hfil\(\displaystyle{##}\)&&\(%
         \displaystyle{{}##}\)\hfil\crcr#1\crcr%
      }
   }
   \,
}

% +--------------------------------------------------------------------+
% |  Hours:minutes macro                                               |
% +--------------------------------------------------------------------+
%\newcount\hrs\newcount\minu\newcount\temptime
%\def\hm{\hrs=\time \divide\hrs by 60 \minu=\time\temptime=\hrs
%\multiply\temptime by 60%
%\advance\minu by -\temptime
%\ifnum\minu<10 \let\zerofill=0\else \let\zerofill=\relax\fi
% \the\hrs:\zerofill\the\minu}



\noindent A length \Macro{figwidth} is defined that is \SI{2}{\cm} smaller than \Macro{textwidth}.

\noindent Most Monte Carlo generators also have a form with a suffix \enquote{V}
that allows you to include the version, e.g.
\verb|\PYTHIAV{8}| to produce \PYTHIAV{8} or
\verb|\PYTHIAV{8 (v8.160)}| to produce \PYTHIAV{8 (v8.160)}.

\noindent A generic macro \verb|\twomass| is defined, so that for example
\verb|\twomass{\mu}{\mu}| produces \twomass{\mu}{\mu} and \verb|\twomass{\mu}{e}| produces \twomass{\mu}{e}.

A macro \verb|\dk| is also defined which makes it easier to write down decay chains.
For example
\begin{verbatim}
\[\eqalign{a \to & b+c\\
   & \dk & e+f \\
   && \dk g+h}
\]
\end{verbatim}
produces
\[\eqalign{a \to & b+c\cr
   & \dk & e+f \cr
   && \dk g+h}
\]
Note that \Macro{eqalign} is also redefined in this package so that \Macro{dk} works.

The following macro names have been changed:\\
\verb|\ptsq| \(\to\) \verb|\pTsq|.


\newpage
%-------------------------------------------------------------------------------
\section{\File{atlasxref.sty}}

Turn on including these definitions with the option \Option{xref=true} and off with the option \Option{xref=false}.

\begin{xtabular}{ll}
\end{xtabular}


\noindent The following macros with arguments are also defined:
\begin{xtabular}{ll}
\verb|\App{1}|  & \App{1}\\
\verb|\Eqn{1}|  & \Eqn{1}\\
\verb|\Fig{1}|  & \Fig{1}\\
\verb|\Refn{1}|  & \Refn{1}\\
\verb|\Sect{1}| & \Sect{1}\\
\verb|\Tab{1}|  & \Tab{1}\\
\verb|\Apps{1}{4}| & \Apps{1}{4} \\
\verb|\Eqns{1}{4}| & \Eqns{1}{4} \\
\verb|\Figs{1}{4}| & \Figs{1}{4} \\
\verb|\Refns{1}{4}| & \Refns{1}{4} \\
\verb|\Sects{1}{4}| & \Sects{1}{4} \\
\verb|\Tabs{1}{4}| & \Tabs{1}{4} \\
\verb|\Apprange{1}{4}| & \Apprange{1}{4} \\
\verb|\Eqnrange{1}{4}| & \Eqnrange{1}{4} \\
\verb|\Figrange{1}{4}| & \Figrange{1}{4} \\
\verb|\Refrange{1}{4}| & \Refrange{1}{4} \\
\verb|\Sectrange{1}{4}| & \Sectrange{1}{4} \\
\verb|\Tabrange{1}{4}| & \Tabrange{1}{4}
\end{xtabular}

The idea is that you can adapt these definitions according to your own preferences (or those of a journal).
Note that the macros \Macro{Ref} and \Macro{Refs} were renamed to \Macro{Refn} and \Macro{Refns}
in \Package{atlaslatex} 08-00-00, as \Macro{Ref} is now defined in the \Package{hyperref} package.


\newpage
%-------------------------------------------------------------------------------
\section{\File{atlasbsm.sty}}

Turn on including these definitions with the option \Option{BSM} and off with the option \Option{BSM=false}.

The macro \Macro{susy} simply puts a tilde (\(\tilde{\ }\)) over its argument,
e.g.\ \verb|\susy{q}| produces \susy{q}.

For \susy{q}, \susy{t}, \susy{b}, \slepton, \sel, \smu and
\stau, L and R states are defined; for stop, sbottom and stau also the
light (1) and heavy (2) states.
There are four neutralinos and two charginos defined,
the index number unfortunately needs to be written out completely.
For the charginos the last letter(s) indicate(s) the charge:
\enquote{p} for \(+\), \enquote{m} for \(-\), and \enquote{pm} for \(\pm\).

% Note that this file can be overwritten when atlaslatex is updated.
%
% Copyright (C) 2002-2020 CERN for the benefit of the ATLAS collaboration
%
% +--------------------------------------------------------------------+
% |                                                                    |
% |  New particle stuff                                                |
% |                                                                    |
% +--------------------------------------------------------------------+
%
\newcommand*{\Azero}{\ensuremath{A^0}\xspace}
\newcommand*{\hzero}{\ensuremath{h^0}\xspace}
\newcommand*{\Hzero}{\ensuremath{H^0}\xspace}
\newcommand*{\Hboson}{\ensuremath{H}\xspace}
\newcommand*{\Hplus}{\ensuremath{H^+}\xspace}
\newcommand*{\Hminus}{\ensuremath{H^-}\xspace}
\newcommand*{\Hpm}{\ensuremath{H^{\pm}}\xspace}
\newcommand*{\Hmp}{\ensuremath{H^{\mp}}\xspace}
\newcommand*{\susy}[1]{\ensuremath{\tilde{#1}}\xspace}
\newcommand*{\ggino}{\ensuremath{\mathchoice%
      {\displaystyle\raise.4ex\hbox{\(\displaystyle\tilde\chi\)}}%
         {\textstyle\raise.4ex\hbox{\(\textstyle\tilde\chi\)}}%
       {\scriptstyle\raise.3ex\hbox{\(\scriptstyle\tilde\chi\)}}%
 {\scriptscriptstyle\raise.3ex\hbox{\(\scriptscriptstyle\tilde\chi\)}}}\xspace}

\newcommand*{\chinop}{\ensuremath{\mathchoice%
      {\displaystyle\raise.4ex\hbox{\(\displaystyle\tilde\chi^+\)}}%
         {\textstyle\raise.4ex\hbox{\(\textstyle\tilde\chi^+\)}}%
       {\scriptstyle\raise.3ex\hbox{\(\scriptstyle\tilde\chi^+\)}}%
 {\scriptscriptstyle\raise.3ex\hbox{\(\scriptscriptstyle\tilde\chi^+\)}}}\xspace}
\newcommand*{\chinom}{\ensuremath{\mathchoice%
      {\displaystyle\raise.4ex\hbox{\(\displaystyle\tilde\chi^-\)}}%
         {\textstyle\raise.4ex\hbox{\(\textstyle\tilde\chi^-\)}}%
       {\scriptstyle\raise.3ex\hbox{\(\scriptstyle\tilde\chi^-\)}}%
 {\scriptscriptstyle\raise.3ex\hbox{\(\scriptscriptstyle\tilde\chi^-\)}}}\xspace}
\newcommand*{\chinopm}{\ensuremath{\mathchoice%
      {\displaystyle\raise.4ex\hbox{\(\displaystyle\tilde\chi^\pm\)}}%
         {\textstyle\raise.4ex\hbox{\(\textstyle\tilde\chi^\pm\)}}%
       {\scriptstyle\raise.3ex\hbox{\(\scriptstyle\tilde\chi^\pm\)}}%
 {\scriptscriptstyle\raise.3ex\hbox{\(\scriptscriptstyle\tilde\chi^\pm\)}}}\xspace}
\newcommand*{\chinomp}{\ensuremath{\mathchoice%
      {\displaystyle\raise.4ex\hbox{\(\displaystyle\tilde\chi^\mp\)}}%
         {\textstyle\raise.4ex\hbox{\(\textstyle\tilde\chi^\mp\)}}%
       {\scriptstyle\raise.3ex\hbox{\(\scriptstyle\tilde\chi^\mp\)}}%
 {\scriptscriptstyle\raise.3ex\hbox{\(\scriptscriptstyle\tilde\chi^\mp\)}}}\xspace}

\newcommand*{\chinoonep}{\ensuremath{\mathchoice%
      {\displaystyle\raise.4ex\hbox{\(\displaystyle\tilde\chi^+_1\)}}%
         {\textstyle\raise.4ex\hbox{\(\textstyle\tilde\chi^+_1\)}}%
       {\scriptstyle\raise.3ex\hbox{\(\scriptstyle\tilde\chi^+_1\)}}%
 {\scriptscriptstyle\raise.3ex\hbox{\(\scriptscriptstyle\tilde\chi^+_1\)}}}\xspace}
\newcommand*{\chinoonem}{\ensuremath{\mathchoice%
      {\displaystyle\raise.4ex\hbox{\(\displaystyle\tilde\chi^-_1\)}}%
         {\textstyle\raise.4ex\hbox{\(\textstyle\tilde\chi^-_1\)}}%
       {\scriptstyle\raise.3ex\hbox{\(\scriptstyle\tilde\chi^-_1\)}}%
 {\scriptscriptstyle\raise.3ex\hbox{\(\scriptscriptstyle\tilde\chi^-_1\)}}}\xspace}
\newcommand*{\chinoonepm}{\ensuremath{\mathchoice%
      {\displaystyle\raise.4ex\hbox{\(\displaystyle\tilde\chi^\pm_1\)}}%
         {\textstyle\raise.4ex\hbox{\(\textstyle\tilde\chi^\pm_1\)}}%
       {\scriptstyle\raise.3ex\hbox{\(\scriptstyle\tilde\chi^\pm_1\)}}%
 {\scriptscriptstyle\raise.3ex\hbox{\(\scriptscriptstyle\tilde\chi^\pm_1\)}}}\xspace}

\newcommand*{\chinotwop}{\ensuremath{\mathchoice%
      {\displaystyle\raise.4ex\hbox{\(\displaystyle\tilde\chi^+_2\)}}%
         {\textstyle\raise.4ex\hbox{\(\textstyle\tilde\chi^+_2\)}}%
       {\scriptstyle\raise.3ex\hbox{\(\scriptstyle\tilde\chi^+_2\)}}%
 {\scriptscriptstyle\raise.3ex\hbox{\(\scriptscriptstyle\tilde\chi^+_2\)}}}\xspace}
\newcommand*{\chinotwom}{\ensuremath{\mathchoice%
      {\displaystyle\raise.4ex\hbox{\(\displaystyle\tilde\chi^-_2\)}}%
         {\textstyle\raise.4ex\hbox{\(\textstyle\tilde\chi^-_2\)}}%
       {\scriptstyle\raise.3ex\hbox{\(\scriptstyle\tilde\chi^-_2\)}}%
 {\scriptscriptstyle\raise.3ex\hbox{\(\scriptscriptstyle\tilde\chi^-_2\)}}}\xspace}
\newcommand*{\chinotwopm}{\ensuremath{\mathchoice%
      {\displaystyle\raise.4ex\hbox{\(\displaystyle\tilde\chi^\pm_2\)}}%
         {\textstyle\raise.4ex\hbox{\(\textstyle\tilde\chi^\pm_2\)}}%
       {\scriptstyle\raise.3ex\hbox{\(\scriptstyle\tilde\chi^\pm_2\)}}%
 {\scriptscriptstyle\raise.3ex\hbox{\(\scriptscriptstyle\tilde\chi^\pm_2\)}}}\xspace}

\newcommand*{\nino}{\ensuremath{\mathchoice%
      {\displaystyle\raise.4ex\hbox{\(\displaystyle\tilde\chi^0\)}}%
         {\textstyle\raise.4ex\hbox{\(\textstyle\tilde\chi^0\)}}%
       {\scriptstyle\raise.3ex\hbox{\(\scriptstyle\tilde\chi^0\)}}%
 {\scriptscriptstyle\raise.3ex\hbox{\(\scriptscriptstyle\tilde\chi^0\)}}}\xspace}

\newcommand*{\ninoone}{\ensuremath{\mathchoice%
      {\displaystyle\raise.4ex\hbox{\(\displaystyle\tilde\chi^0_1\)}}%
         {\textstyle\raise.4ex\hbox{\(\textstyle\tilde\chi^0_1\)}}%
       {\scriptstyle\raise.3ex\hbox{\(\scriptstyle\tilde\chi^0_1\)}}%
 {\scriptscriptstyle\raise.3ex\hbox{\(\scriptscriptstyle\tilde\chi^0_1\)}}}\xspace}
\newcommand*{\ninotwo}{\ensuremath{\mathchoice%
      {\displaystyle\raise.4ex\hbox{\(\displaystyle\tilde\chi^0_2\)}}%
         {\textstyle\raise.4ex\hbox{\(\textstyle\tilde\chi^0_2\)}}%
       {\scriptstyle\raise.3ex\hbox{\(\scriptstyle\tilde\chi^0_2\)}}%
 {\scriptscriptstyle\raise.3ex\hbox{\(\scriptscriptstyle\tilde\chi^0_2\)}}}\xspace}
\newcommand*{\ninothree}{\ensuremath{\mathchoice%
      {\displaystyle\raise.4ex\hbox{\(\displaystyle\tilde\chi^0_3\)}}%
         {\textstyle\raise.4ex\hbox{\(\textstyle\tilde\chi^0_3\)}}%
       {\scriptstyle\raise.3ex\hbox{\(\scriptstyle\tilde\chi^0_3\)}}%
 {\scriptscriptstyle\raise.3ex\hbox{\(\scriptscriptstyle\tilde\chi^0_3\)}}}\xspace}
\newcommand*{\ninofour}{\ensuremath{\mathchoice%
      {\displaystyle\raise.4ex\hbox{\(\displaystyle\tilde\chi^0_4\)}}%
         {\textstyle\raise.4ex\hbox{\(\textstyle\tilde\chi^0_4\)}}%
       {\scriptstyle\raise.3ex\hbox{\(\scriptstyle\tilde\chi^0_4\)}}%
 {\scriptscriptstyle\raise.3ex\hbox{\(\scriptscriptstyle\tilde\chi^0_4\)}}}\xspace}

\newcommand*{\gravino}{\ensuremath{\tilde{G}}\xspace}
\newcommand*{\Zprime}{\ensuremath{Z^\prime}\xspace}
\newcommand*{\Zstar}{\ensuremath{Z^{*}}\xspace}
\newcommand*{\squark}{\ensuremath{\tilde{q}}\xspace}
\newcommand*{\squarkL}{\ensuremath{\tilde{q}_{\mathrm{L}}}\xspace}
\newcommand*{\squarkR}{\ensuremath{\tilde{q}_{\mathrm{R}}}\xspace}
\newcommand*{\gluino}{\ensuremath{\tilde{g}}\xspace}
\renewcommand*{\stop}{\ensuremath{\tilde{t}}\xspace}
\newcommand*{\stopone}{\ensuremath{\tilde{t}_1}\xspace}
\newcommand*{\stoptwo}{\ensuremath{\tilde{t}_2}\xspace}
\newcommand*{\stopL}{\ensuremath{\tilde{t}_{\mathrm{L}}}\xspace}
\newcommand*{\stopR}{\ensuremath{\tilde{t}_{\mathrm{R}}}\xspace}
\newcommand*{\sbottom}{\ensuremath{\tilde{b}}\xspace}
\newcommand*{\sbottomone}{\ensuremath{\tilde{b}_1}\xspace}
\newcommand*{\sbottomtwo}{\ensuremath{\tilde{b}_2}\xspace}
\newcommand*{\sbottomL}{\ensuremath{\tilde{b}_{\mathrm{L}}}\xspace}
\newcommand*{\sbottomR}{\ensuremath{\tilde{b}_{\mathrm{R}}}\xspace}
\newcommand*{\slepton}{\ensuremath{\tilde{\ell}}\xspace}
\newcommand*{\sleptonL}{\ensuremath{\tilde{\ell}_{\mathrm{L}}}\xspace}
\newcommand*{\sleptonR}{\ensuremath{\tilde{\ell}_{\mathrm{R}}}\xspace}
\newcommand*{\sel}{\ensuremath{\tilde{e}}\xspace}
\newcommand*{\selL}{\ensuremath{\tilde{e}_{\mathrm{L}}}\xspace}
\newcommand*{\selR}{\ensuremath{\tilde{e}_{\mathrm{R}}}\xspace}
\newcommand*{\smu}{\ensuremath{\tilde{\mu}}\xspace}
\newcommand*{\smuL}{\ensuremath{\tilde{\mu}_{\mathrm{L}}}\xspace}
\newcommand*{\smuR}{\ensuremath{\tilde{\mu}_{\mathrm{R}}}\xspace}
\newcommand*{\stau}{\ensuremath{\tilde{\tau}}\xspace}
\newcommand*{\stauL}{\ensuremath{\tilde{\tau}_{\mathrm{L}}}\xspace}
\newcommand*{\stauR}{\ensuremath{\tilde{\tau}_{\mathrm{R}}}\xspace}
\newcommand*{\stauone}{\ensuremath{\tilde{\tau}_1}\xspace}
\newcommand*{\stautwo}{\ensuremath{\tilde{\tau}_2}\xspace}
\newcommand*{\snu}{\ensuremath{\tilde{\nu}}\xspace}



\newpage
%-------------------------------------------------------------------------------
\section{\File{atlasheavyion.sty}}

Turn on including these definitions with the option \Option{hion=true} and off with the option \Option{hion=false}.
The heavy ion definitions use the package \Package{mhchem} to help with the formatting of chemical elements.
This package is included by \File{atlasheavyion.sty}.

%-------------------------------------------------------------------------------
% Collection of heavy iondefinitions, typically not included in the other style files.
% Include with hion option in atlasphysics.sty.
% Not included by default.
% Also needs atlasmisc.sty (option misc).
% Compiled by Sasha Milov.
% Adapted for atlaslatex by Ian Brock.
%
% Note that this file can be overwritten when atlaslatex is updated.
%
% Copyright (C) 2002-2020 CERN for the benefit of the ATLAS collaboration
%-------------------------------------------------------------------------------

% Package used for chemical elements
\RequirePackage[version=3]{mhchem}

% +------------------------------------+
% |                                    |
% |  System related notations          |
% |                                    |
% +------------------------------------+
\newcommand*{\NucNuc}{\ce{A}+\ce{A}\xspace}

\newcommand*{\nn}{\ensuremath{nn}\xspace}
%\newcommand*{\pp}{\ensuremath{pp}\xspace}
\newcommand*{\pn}{\ensuremath{pn}\xspace}
\newcommand*{\np}{\ensuremath{np}\xspace}

\newcommand*{\PbPb}{\ce{Pb}+\ce{Pb}\xspace}
\newcommand*{\AuAu}{\ce{Au}+\ce{Au}\xspace}
\newcommand*{\CuCu}{\ce{Cu}+\ce{Cu}\xspace}

\providecommand*{\pA}{\ensuremath{p}+\ce{A}\xspace}
\newcommand*{\pNuc}{\pA\xspace}
\newcommand*{\pdA}{\ensuremath{p}/\ensuremath{d}+\ce{A}\xspace}
\newcommand*{\dAu}{\ensuremath{d}+\ce{Au}\xspace}
\newcommand*{\pPb}{\ensuremath{p}+\ce{Pb}\xspace}

% +--------------------------------------+
% |                                      |
% |  Centrality related notations        |
% |                                      |
% +--------------------------------------+
\newcommand*{\Npart}{\ensuremath{N_{\text{part}}}\xspace}
\newcommand*{\avgNpart}{\ensuremath{\langle\Npart\rangle}\xspace}

\newcommand*{\Ncoll}{\ensuremath{N_{\text{coll}}}\xspace}
\newcommand*{\avgNcoll}{\ensuremath{\langle\Ncoll\rangle}\xspace}

\newcommand*{\TA}{\ensuremath{T_{\ce{A}}}\xspace}
\newcommand*{\avgTA}{\ensuremath{\langle\TA\rangle}\xspace}

\newcommand*{\TPb}{\ensuremath{T_{\ce{Pb}}}\xspace}
\newcommand*{\avgTPb}{\ensuremath{\langle\TPb\rangle}\xspace}

\newcommand*{\TAA}{\ensuremath{T_{\text{AA}}}\xspace}
\newcommand*{\avgTAA}{\ensuremath{\langle\TAA\rangle}\xspace}

\newcommand{\TAB}{\ensuremath{T_{\text{AB}}}\xspace}
\newcommand{\avgTAB}{\ensuremath{\langle\TAB\rangle}\xspace}

\newcommand*{\TpPb}{\ensuremath{T_{p\ce{Pb}}}\xspace}
\newcommand*{\avgTpPb}{\ensuremath{\langle\TpPb\rangle}\xspace}

\newcommand{\Gl}{Glauber\xspace}
\newcommand{\GG}{Glauber--Gribov\xspace}

% +--------------------------------------+
% |                                      |
% |  C.M. energy related notations       |
% |                                      |
% +--------------------------------------+
\newcommand*{\sqn}{\ensuremath{\sqrt{s_{_\text{NN}}}}\xspace}
\newcommand{\lns}{\ensuremath{\ln(\kern -0.2em\sqrt{s})}\xspace}

% +--------------------------------------+
% |                                      |
% |  Some useful parameters              |
% |                                      |
% +--------------------------------------+
\newcommand*{\sumETPb}{\ensuremath{\Sigma E_{\text{T}}^{\ce{Pb}}}\xspace}
\newcommand*{\sumETp}{\ensuremath{\Sigma E_{\text{T}}^{p}}\xspace}
\newcommand*{\sumETA}{\ensuremath{\Sigma E_{\text{T}}^{\ce{A}}}\xspace}

% +--------------------------------------+
% |                                      |
% |  Some useful constructions           |
% |                                      |
% +--------------------------------------+
\newcommand*{\RAA}{\ensuremath{R_{\ce{AA}}}\xspace}
\newcommand*{\RCP}{\ensuremath{R_{\text{CP}}}\xspace}
\newcommand*{\RpA}{\ensuremath{R_{p\ce{A}}}\xspace}

\newcommand*{\RpPb}{\ensuremath{R_{p\ce{Pb}}}\xspace}

% Different differential symbols in American and British English
\iflanguage{USenglish}{%
  \providecommand*{\dif}{\ensuremath{d}}
}{%
  \providecommand*{\dif}{\ensuremath{\mathrm{d}}}
}
\newcommand*{\dNchdeta}{\ensuremath{\dif N_{\text{ch}}/\dif \eta}\xspace}
\newcommand*{\dNevtdET}{\ensuremath{\dif N_{\text{evt}}/\dif \ET}\xspace}

% +--------------------------------------+
% |                                      |
% |  Framework transforms                |
% |                                      |
% +--------------------------------------+
\newcommand*{\ystar}{\ensuremath{y^{*}}\xspace}
\newcommand*{\ycms}{\ensuremath{y_\text{CM}}\xspace}
\newcommand*{\ygappb}{\ensuremath{\Delta \eta_{\text{gap}}^{\ce{Pb}}}\xspace}
\newcommand*{\ygapp}{\ensuremath{\Delta \eta_{\text{gap}}^{p}}\xspace}
\newcommand*{\fgap}{\ensuremath{f_{\text{gap}}}\xspace}


%The following symbols were removed or modified with respect to the original submission
%\input{atlasheavyion-mod}


\newpage
%-------------------------------------------------------------------------------
\section{\File{atlasjetetmiss.sty}}

Turn on including these definitions with the option \Option{jetetmiss=true} and off with the option \Option{jetetmiss=false}.

\begin{xtabular}{ll}
\verb|\topo| & \topo \\
\verb|\Topo| & \Topo \\
\verb|\topos| & \topos \\
\verb|\Topos| & \Topos \\
\verb|\insitu| & \insitu \\
\verb|\Insitu| & \Insitu \\
\verb|\LS| & \LS \\
\verb|\NLOjet| & \NLOjet \\
\verb|\Fastjet| & \Fastjet \\
\verb|\TwoToTwo| & \TwoToTwo \\
\verb|\largeR| & \largeR \\
\verb|\LargeR| & \LargeR \\
\verb|\akt| & \akt \\
\verb|\Akt| & \Akt \\
\verb|\AKT| & \AKT \\
\verb|\AKTFat| & \AKTFat \\
\verb|\AKTPrune| & \AKTPrune \\
\verb|\AKTFilt| & \AKTFilt \\
\verb|\KTSix| & \KTSix \\
\verb|\ca| & \ca \\
\verb|\CamKt| & \CamKt \\
\verb|\CASix| & \CASix \\
\verb|\CAFat| & \CAFat \\
\verb|\CAPrune| & \CAPrune \\
\verb|\CAFilt| & \CAFilt \\
\verb|\htt| & \htt \\
\verb|\mcut| & \mcut \\
\verb|\Nfilt| & \Nfilt \\
\verb|\Rfilt| & \Rfilt \\
\verb|\ymin| & \ymin \\
\verb|\fcut| & \fcut \\
\verb|\Rsub| & \Rsub \\
\verb|\mufrac| & \mufrac \\
\verb|\Rcut| & \Rcut \\
\verb|\zcut| & \zcut \\
\verb|\ftile| & \ftile \\
\verb|\fem| & \fem \\
\verb|\fpres| & \fpres \\
\verb|\fhec| & \fhec \\
\verb|\ffcal| & \ffcal \\
\verb|\central| & \central \\
\verb|\ecap| & \ecap \\
\verb|\forward| & \forward \\
\verb|\Npv| & \Npv \\
\verb|\Nref| & \Nref \\
\verb|\Navg| & \Navg \\
\verb|\avgmu| & \avgmu \\
\verb|\JES| & \JES \\
\verb|\JMS| & \JMS \\
\verb|\EMJES| & \EMJES \\
\verb|\GCWJES| & \GCWJES \\
\verb|\LCWJES| & \LCWJES \\
\verb|\EM| & \EM \\
\verb|\GCW| & \GCW \\
\verb|\LCW| & \LCW \\
\verb|\GSL| & \GSL \\
\verb|\GS| & \GS \\
\verb|\MTF| & \MTF \\
\verb|\MPF| & \MPF \\
\verb|\Njet| & \Njet \\
\verb|\njet| & \njet \\
\verb|\ETjet| & \ETjet \\
\verb|\etjet| & \etjet \\
\verb|\pTavg| & \pTavg \\
\verb|\ptavg| & \ptavg \\
\verb|\pTjet| & \pTjet \\
\verb|\ptjet| & \ptjet \\
\verb|\pTcorr| & \pTcorr \\
\verb|\ptcorr| & \ptcorr \\
\verb|\pTjeti| & \pTjeti \\
\verb|\ptjeti| & \ptjeti \\
\verb|\pTrecoil| & \pTrecoil \\
\verb|\ptrecoil| & \ptrecoil \\
\verb|\pTleading| & \pTleading \\
\verb|\ptleading| & \ptleading \\
\verb|\pTjetEM| & \pTjetEM \\
\verb|\ptjetEM| & \ptjetEM \\
\verb|\pThat| & \pThat \\
\verb|\pthat| & \pthat \\
\verb|\pTprobe| & \pTprobe \\
\verb|\ptprobe| & \ptprobe \\
\verb|\pTref| & \pTref \\
\verb|\ptref| & \ptref \\
\verb|\pToff| & \pToff \\
\verb|\ptoff| & \ptoff \\
\verb|\pToffjet| & \pToffjet \\
\verb|\ptoffjet| & \ptoffjet \\
\verb|\pTZ| & \pTZ \\
\verb|\ptZ| & \ptZ \\
\verb|\pTtrue| & \pTtrue \\
\verb|\pttrue| & \pttrue \\
\verb|\pTtruth| & \pTtruth \\
\verb|\pttruth| & \pttruth \\
\verb|\pTreco| & \pTreco \\
\verb|\ptreco| & \ptreco \\
\verb|\pTtrk| & \pTtrk \\
\verb|\pttrk| & \pttrk \\
\verb|\ptrk| & \ptrk \\
\verb|\pTtrkjet| & \pTtrkjet \\
\verb|\pttrkjet| & \pttrkjet \\
\verb|\ntrk| & \ntrk \\
\verb|\EoverP| & \EoverP \\
\verb|\Etrue| & \Etrue \\
\verb|\Etruth| & \Etruth \\
\verb|\Ecalo| & \Ecalo \\
\verb|\EcaloEM| & \EcaloEM \\
\verb|\asym| & \asym \\
\verb|\Response| & \Response \\
\verb|\Rcalo| & \Rcalo \\
\verb|\Rcalom| & \Rcalom \\
\verb|\RcaloEM| & \RcaloEM \\
\verb|\RMPF| & \RMPF \\
\verb|\EcaloCALIB| & \EcaloCALIB \\
\verb|\RcaloCALIB| & \RcaloCALIB \\
\verb|\EcaloEMJES| & \EcaloEMJES \\
\verb|\RcaloEMJES| & \RcaloEMJES \\
\verb|\EcaloGCWJES| & \EcaloGCWJES \\
\verb|\RcaloGCWJES| & \RcaloGCWJES \\
\verb|\EcaloLCWJES| & \EcaloLCWJES \\
\verb|\RcaloLCWJES| & \RcaloLCWJES \\
\verb|\Rtrack| & \Rtrack \\
\verb|\rtrk| & \rtrk \\
\verb|\Rtrk| & \Rtrk \\
\verb|\rtrackjet| & \rtrackjet \\
\verb|\rtrackjetiso| & \rtrackjetiso \\
\verb|\rtrackjetnoniso| & \rtrackjetnoniso \\
\verb|\rtrackjetisoratio| & \rtrackjetisoratio \\
\verb|\gammajet| & \gammajet \\
\verb|\deltaphijetgamma| & \deltaphijetgamma \\
\verb|\rapjet| & \rapjet \\
\verb|\etajet| & \etajet \\
\verb|\phijet| & \phijet \\
\verb|\etadet| & \etadet \\
\verb|\etatrk| & \etatrk \\
\verb|\Rmin| & \Rmin \\
\verb|\DeltaR| & \DeltaR \\
\verb|\DetaDphi| & \DetaDphi \\
\verb|\Deta| & \Deta \\
\verb|\Drap| & \Drap \\
\verb|\DetaOneTwo| & \DetaOneTwo \\
\verb|\DyDphi| & \DyDphi \\
\verb|\DeltaRdef| & \DeltaRdef \\
\verb|\DeltaRydef| & \DeltaRydef \\
\verb|\DeltaRtrk| & \DeltaRtrk \\
\verb|\JVF| & \JVF \\
\verb|\cJVF| & \cJVF \\
\verb|\RpT| & \RpT \\
\verb|\JVT| & \JVT \\
\verb|\ghostpt| & \ghostpt \\
\verb|\ghostptavg| & \ghostptavg \\
\verb|\ghostfm| & \ghostfm \\
\verb|\ghostfmi| & \ghostfmi \\
\verb|\ghostdensity| & \ghostdensity \\
\verb|\ghostrho| & \ghostrho \\
\verb|\Aghost| & \Aghost \\
\verb|\Amu| & \Amu \\
\verb|\Amui| & \Amui \\
\verb|\jetarea| & \jetarea \\
\verb|\jetareafm| & \jetareafm \\
\verb|\jetareai| & \jetareai \\
\verb|\Rkt| & \Rkt \\
\verb|\pTmuslope| & \pTmuslope \\
\verb|\ptmuslope| & \ptmuslope \\
\verb|\pTnpvslope| & \pTnpvslope \\
\verb|\ptnpvslope| & \ptnpvslope \\
\verb|\pTmuunc| & \pTmuunc \\
\verb|\ptmuunc| & \ptmuunc \\
\verb|\pTnpvunc| & \pTnpvunc \\
\verb|\ptnpvunc| & \ptnpvunc \\
\verb|\sumPt| & \sumPt \\
\verb|\sumpt| & \sumpt \\
\verb|\sumpTtrk| & \sumpTtrk \\
\verb|\sumpttrk| & \sumpttrk \\
\verb|\nPUtrk| & \nPUtrk \\
\verb|\mjet| & \mjet \\
\verb|\mlead| & \mlead \\
\verb|\mleadavg| & \mleadavg \\
\verb|\Mjet| & \Mjet \\
\verb|\massjet| & \massjet \\
\verb|\masscorr| & \masscorr \\
\verb|\mthresh| & \mthresh \\
\verb|\mjetavg| & \mjetavg \\
\verb|\masstrkjet| & \masstrkjet \\
\verb|\width| & \width \\
\verb|\wcalo| & \wcalo \\
\verb|\wtrk| & \wtrk \\
\verb|\shapeV| & \shapeV \\
\verb|\pTsubjet| & \pTsubjet \\
\verb|\ptsubjet| & \ptsubjet \\
\verb|\sjone| & \sjone \\
\verb|\sjtwo| & \sjtwo \\
\verb|\msubjone| & \msubjone \\
\verb|\msubjtwo| & \msubjtwo \\
\verb|\pTsubji| & \pTsubji \\
\verb|\ptsubji| & \ptsubji \\
\verb|\pTsubjone| & \pTsubjone \\
\verb|\ptsubjone| & \ptsubjone \\
\verb|\pTsubjtwo| & \pTsubjtwo \\
\verb|\ptsubjtwo| & \ptsubjtwo \\
\verb|\Rsubjets| & \Rsubjets \\
\verb|\DRsubjets| & \DRsubjets \\
\verb|\yij| & \yij \\
\verb|\dcut| & \dcut \\
\verb|\dmin| & \dmin \\
\verb|\dij| & \dij \\
\verb|\Dij| & \Dij \\
\verb|\Donetwo| & \Donetwo \\
\verb|\Dtwothr| & \Dtwothr \\
\verb|\yonetwo| & \yonetwo \\
\verb|\ytwothr| & \ytwothr \\
\verb|\yonetwoDef| & \yonetwoDef \\
\verb|\ytwothrDef| & \ytwothrDef \\
\verb|\xj| & \xj \\
\verb|\jetFunc| & \jetFunc \\
\verb|\tauone| & \tauone \\
\verb|\tautwo| & \tautwo \\
\verb|\tauthr| & \tauthr \\
\verb|\tauN| & \tauN \\
\verb|\tautwoone| & \tautwoone \\
\verb|\tauthrtwo| & \tauthrtwo \\
\verb|\dip| & \dip \\
\verb|\diponetwo| & \diponetwo \\
\verb|\diptwothr| & \diptwothr \\
\verb|\diponethr| & \diponethr \\
\verb|\mtaSup| & \mtaSup \\
\verb|\mcalo| & \mcalo \\
\verb|\mcomb| & \mcomb \\
\verb|\ECFOne| & \ECFOne \\
\verb|\ECFTwo| & \ECFTwo \\
\verb|\ECFThr| & \ECFThr \\
\verb|\ECFThrNorm| & \ECFThrNorm \\
\verb|\DTwo| & \DTwo \\
\verb|\CTwo| & \CTwo \\
\verb|\FoxWolfRatio| & \FoxWolfRatio \\
\verb|\PlanarFlow| & \PlanarFlow \\
\verb|\Angularity| & \Angularity \\
\verb|\Aplanarity| & \Aplanarity \\
\verb|\KtDR| & \KtDR \\
\verb|\Qw| & \Qw \\
\verb|\NConst| & \NConst \\
\end{xtabular}


\noindent The macro \Macro{etaRange} produces what you would expect:
\verb|\etaRange{-2.5}{+2.5}| produces \etaRange{-2.5}{+2.5} while
\verb|\AetaRange{1.0}| produces \AetaRange{1.0}.
The macro \Macro{avg} can be used for average values:
\verb|\avg{\mu}| produces \avg{\mu}.


\newpage
%-------------------------------------------------------------------------------
\section{\File{atlasmath.sty}}

Turn on including these definitions with the option \Option{math=true} and off with the option \Option{math=false}.

\begin{xtabular}{ll}
\verb|\boxsq| & \boxsq \\
\verb|\grad| & \grad \\
\end{xtabular}


\noindent The macro \Macro{spinor} is also defined.
\verb|\spinor{u}| produces \spinor{u}.


\newpage
%-------------------------------------------------------------------------------
\section{\File{atlasother.sty}}

Turn on including these definitions with the option \Option{other} and off with the option \Option{other=false}.

\begin{xtabular}{ll}
\verb|\etpt| & \etpt \\
\verb|\etptsig| & \etptsig \\
\verb|\begL| & \begL \\
\verb|\lowL| & \lowL \\
\verb|\highL| & \highL \\
\verb|\Epsb| & \Epsb \\
\verb|\Epsc| & \Epsc \\
\verb|\Mtau| & \Mtau \\
\verb|\swsq| & \swsq \\
\verb|\swel| & \swel \\
\verb|\swsqb| & \swsqb \\
\verb|\swsqon| & \swsqon \\
\verb|\gv| & \gv \\
\verb|\ga| & \ga \\
\verb|\gvbar| & \gvbar \\
\verb|\gabar| & \gabar \\
\verb|\Zzv| & \Zzv \\
\verb|\Abb| & \Abb \\
\verb|\Acc| & \Acc \\
\verb|\Aqq| & \Aqq \\
\verb|\Afb| & \Afb \\
\verb|\GZ| & \GZ \\
\verb|\GW| & \GW \\
\verb|\GH| & \GH \\
\verb|\GamHad| & \GamHad \\
\verb|\Gbb| & \Gbb \\
\verb|\Rbb| & \Rbb \\
\verb|\Gcc| & \Gcc \\
\verb|\Gvis| & \Gvis \\
\verb|\Ginv| & \Ginv \\
\end{xtabular}

